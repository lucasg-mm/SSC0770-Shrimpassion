%! TEX root = **/000-main.tex
% vim: spell spelllang=en:

\section{Visão Geral}%
    \subsection{Tema/Contexto/Genêro}%
    A tematica principal do jogo será a de frutos do mar, sendo usados tanto elementos biologia/vida marinha quanto de seus usos culinarios, para trazer essa ideia a vida sera usado o genero de platformer com elementos de ação, sendo um tradicional "corre e atira" 

    \subsection{Mecânica Central}%
- Movimentação:\\
    Andar, saltar, dash (maybe escalada ???)\\
- Combate:\\
    O protagonista tem 2 tipos de ataques, um melee, usando suas pinças e um ranged que é um disparo de agua, sendo esse disparo limitado a munição, agua, que o protagonista possui \\
    (maybe upgrade de habilidades)\\

    \subsection{Plataformas Alvo}%
Inicialmente PC, windows e linux com possibilidade de expanção para consoles.
    \subsection{Público Alvo}%
Publico jovem, gen z, que gostam de um humor nonsense (o protagonista é um CAMARÃO) mas que não abrem mão da experiencia de um bom platformer.
    \subsection{Escopo do Projeto}%
Tempo de desenvolvimento: aproximadamente 3 meses.
Equipe:
    TODO inserir lista com os nomes dos membros e cargos
    \subsection{Influências}%
Megaman:\\ 
TODO inserir imagens do jogo\\
Cup Head:\\
TODO inserir imagens do jogo\\

Biologia:\\
TODO inserir imagem do camarão pistola\

    \subsection{Descrição do projeto}%
Em um cup noodles de frutos do mar, o unico camarão desidratado do pacote, solitario e na escuridão, tenta escapar de sua prisão, para voltar ao mar em busca de seu amor.\\
Embarque nessa jornada cheia de perigos e plataformas, usando sua poderosa pinça para esmagar inimigos que ficarem entre seu você e o amor verdadeiro, e se não ousarem chegar basta bater suas pinças para criar um pulso de agua capaz de destruir qualquer um.\\
Nada pode parar um camarão determinado\\
    \subsection{Descrição do projeto (detalhado)}%
TODO: detalhar a sessão anterior com um mais da lore.\\
    \subsection{Mockup das telas}%
    \subsection{Mecanicas centrais}%
    \subsection{Controles}%
    
\section{Historia/Lore}%
    \subsection{Historia Breve}%
    \subsection{Historia detalhada}%

\section{Gameplay e fluxo de jogo}%
    \subsection{Gameplay breve}%
    \subsection{Gameplay detalhado}%


